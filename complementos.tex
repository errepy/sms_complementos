\documentclass[letterpaper]{article}
\usepackage[utf8]{inputenc}
\usepackage[spanish]{babel}
\usepackage[pdftex]{graphicx}
\usepackage{amssymb}
\usepackage{amsmath}
\usepackage{url}
\usepackage{slashbox}


\usepackage[pdftitle={complementos de informatica},
			pdfauthor={Insaurralde},
			pdfkeywords={ paraguay}
		   ]{hyperref}
		   
\hypersetup{
    colorlinks,
    citecolor=black,
    filecolor=black,
    linkcolor=black,
    urlcolor=black
}

\newcommand{\HRule}{\rule{\linewidth}{0.5mm}}

\setcounter{tocdepth}{3}
\makeatletter

\title{Complementos de Informatica}
\author{Insaurralde}

\makeatletter
\g@addto@macro{\UrlBreaks}{\UrlOrds}
\makeatother

\begin{document}

	\begin{titlepage}
		\begin{center}
			\includegraphics[width=0.5\textwidth]{img/logo_uca.jpg}\\[1cm]    
			\textsc{\LARGE Complementos de Informatica}\\[1.5cm]


			% Titulo
			\HRule \\[0.4cm]
			{ \huge \bfseries Applying model-driven engineering in small software
enterprises }\\[0.4cm]
			\HRule \\[0.4cm]

			% Autor y Supervisor
			\begin{minipage}{0.4\textwidth}
				\begin{flushleft} \large
					\emph{Autor:} \\ \textsc{Ramon Insaurralde}
				\end{flushleft}
			\end{minipage}
		
			\vfill
		\begin{abstract}
				Este trabajo esta orientado a identificar los conceptos principales referentes al crowdsourcing y formas de aplicación asociados a este metodo. El crowdsourcing prentende a traves de la globalización llegar a mas personas con capacidades en diferentes áreas e impulsar el Comunity Working para resolver problemas.	             
		
			\end{abstract}

		\end{center}
	\end{titlepage}

	\tableofcontents
	\newpage
	\section{Introducción}

		
	
\section{Marco teórico}


\section{Planeamiento}

\section{Análisis de los resultados}

\section{Discusiones finales}

\section{Conclusiones}







		\newpage
	\section{Bibliografía}
		\bibliography{bibliografia}
		\bibliographystyle{ieeetr}
		\nocite{*}

\end{document}








